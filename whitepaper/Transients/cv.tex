% ====================================================================
%+
% SECTION:
%    cv.tex
%
% CHAPTER:
%    transients.tex
%
% ELEVATOR PITCH:
%    Explain in a few sentences what the relevant discovery or
%    measurement is going to be discussed, and what will be important
%    about it. This is for the browsing reader to get a quick feel
%    for what this section is about.
%
% AUTHORS:
%    Federica Bianco (@fedhere)
%
% ====================================================================

% \section{Cataclismic Variables}
\subsection{Cataclismic Variables}
\def\secname{\chpname:CVtransients}\label{sec:\secname}

\credit{paulaszkody},
\credit{fedhere}

Cataclysmic Variables (CVs) encompass a broad group of objects
including novae, dwarf novae, novalikes, and AM CVn systems, all with different
amplitudes and rate of variability. The one thing they all have in
common is active mass transfer from a late type companion to a
white dwarf. These create variability on a wide range of timescales:

\begin{itemize}
	\item \textit{minutes} flickering in
dwarf novae and novalikes, pulsations in accreting white dwarfs in
the instability strip, orbital periods of AM CVn systems
\item \textit{hours} orbital periods of novae, dwarf novae and novalikes
\item \textit{days} normal outburst lengths of dwarf novae
\item \textit{weeks} outburst length of superoutbursts in short orbital period
dwarf novae, outburst recurrence time of normal outbursts in short
orbital period dwarf novae
\item \textit{months} outburst recurrence time of
longer period dwarf novae, various state changes in novalikes, declines
in novae
\item \textit{years} for the outburst recurrence timescales of the
shortest period dwarf novae and the recurrence times in recurrent novae
\end{itemize}
The
amplitudes range from tenths of mags for flickering and pulsations to 4 mags
for normal dwarf novae and changes in novalike states up to 9-15 mags for the
largest amplitude dwarf novae and classical novae.

These large differences make correct classification with LSST difficult
but necessary in order to reach goals of assessing the correct number
of types of objects for population studies of the end points of
binary evolution. Multiple filters (especially the blue $u$ and $g$)
along with amplitude and recurrence of variation provide the best
discrimination, as all CVs are bluer during outburst and high states of
accretion. Long term, evenly sampled observations can provide indications
of the low amplitude random variability and catch some of the more frequent
outbursts, but higher sampling is needed to determine whether an object
has a normal or superoutburst, to catch a rise to outburst or to a
different accretion state or to follow a nova. Novae typically
have rise times of a few days, while the decline time and shape provide
information as to the mass, distance and composition. The time to decline
by 2-3 magnitudes is correlated with composition,
%
% FED: what is a range of time scales for this decline? days? months?
%
WD mass and location in
the galaxy, thus enabling a study of Galactic chemical evolution.  As with SN,
the diagnostic power for all these systems rests on color and sampling.

Metrics to be developed would assess the abilities of  a given observing
strategy to distinguish between new novae and dwarf novae outbursts and
identify high and low states.  This discriminiation is provided by
measurement of the shapes and recurrence times of large variations as well
as blue colors to distinguish low amplitude variability that would indicate
new pulsators or novalikes. Population studies rely on the numbers of long
orbital period (low amplitude, wide outbursts) vs. short orbital period
(patterns of short outbursts followed by larger, longer superoutbursts)
dwarf novae at different places in the galaxy, as well as the numbers of
recurrent (1-10 yrs) vs. normal novae (10,000 yrs, about 35/galaxy/yr).
Objects particulary worthy of later followup are containing highly magnetic
white dwarfs. These objects can be identified in a large sample when the
magnitude for a majority of the years is a faint (low) state and a small
percentage of time is a bright (high) state, combined with a red color (due
to cyclotron emission from the magnetic accretion column).

% ====================================================================
%
\subsection{Conclusions}
%
% Here we answer the ten questions posed in
% \autoref{sec:intro:evaluation:caseConclusions}:
%
 \begin{description}

 \item[Q1:] {\it Does the science case place any constraints on the
 tradeoff between the sky coverage and coadded depth? For example, should
 the sky coverage be maximized (to $\sim$30,000 deg$^2$, as e.g., in
 Pan-STARRS) or the number of detected galaxies (the current baseline
 of 18,000 deg$^2$)?}

 \item[A1:] Extending the sky area is not a priority for this science.

 \item[Q2:] {\it Does the science case place any constraints on the
 tradeoff between uniformity of sampling and frequency of  sampling? For
 example, a rolling cadence can provide enhanced sample rates over a part
 of the survey or the entire survey for a designated time at the cost of
 reduced sample rate the rest of the time (while maintaining the nominal
 total visit counts).}

 \item[A2:] Intervals of higher cadence are extremely valuable and a rolling cadence is a
satisfactory approach, subject to cadence details.

 \item[Q3:] {\it Does the science case place any constraints on the
 tradeoff between the single-visit depth and the number of visits
 (especially in the $u$-band where longer exposures would minimize the
 impact of the readout noise)?}

 \item[A3:] Increasing the number of $u$-band visits will improve the characterization of CV phenomena.

 \item[Q4:] {\it Does the science case place any constraints on the
 Galactic plane coverage (spatial coverage, temporal sampling, visits per
 band)?}

 \item[A4:] Most CVs will be detected in the galactic plane, and a long, rich series of visits is needed.

 \item[Q5:] {\it Does the science case place any constraints on the
 fraction of observing time allocated to each band?}

 \item[A5:] $u$-band is diagnostic, especially $u$-$g$.

 \item[Q6:] {\it Does the science case place any constraints on the
 cadence for deep drilling fields?}

 \item[A6:] For deep drilling in the galactic plane or for local group galaxies, the best cadences would obtain several epochs  per night in
 each filter, rather than concentrating all acquisition with a filter in a single rapid burst.

 \item[Q7:] {\it Assuming two visits per night, would the science case
 benefit if they are obtained in the same band or not?}

 \item[A7:] Same filter and different filter each offer valuable information, and a mix of these two options would be preferred
pending test of both cadences.

 \item[Q8:] {\it Will the case science benefit from a special cadence
 prescription during commissioning or early in the survey, such as:
 acquiring a full 10-year count of visits for a small area (either in all
 the bands or in a  selected set); a greatly enhanced cadence for a small
 area?}

 \item[A8:] It would be very helpful to CV studies - and many other areas of transient science - to understand variability across all timescales.
 Especially valuable would be a cadence that would cover one (cluster, rich star field) with all the timescales that will not be
strongly represented  in the main survey, starting at 15 seconds.

 \item[Q9:] {\it Does the science case place any constraints on the
 sampling of observing conditions (e.g., seeing, dark sky, airmass),
 possibly as a function of band, etc.?}

 \item[A9:] No.

 \item[Q10:] {\it Does the case have science drivers that would require
 real-time exposure time optimization to obtain nearly constant
 single-visit limiting depth?}

 \item[A10:] No.

 \end{description}

 \navigationbar
