% ====================================================================

\chapter{The Magellanic Clouds}
\def\chpname{MCs}\label{chp:\chpname}

Chapter Editors:
\credit{dnidever},
\credit{knutago}

% \section*{Summary}
% \addcontentsline{toc}{section}{~~~~~~~~~Summary}
%
% Executive summary goes here, highlighting the primary conclusions from
% the chapter's science cases. This should be abstract length, no more:
% say, 200 words.

\section{Introduction}

The Magellanic Clouds have always had outsized importance for
astrophysics.  They are critical steps in the cosmological distance
ladder, they are a binary galaxy system with a unique interaction
history, and they are laboratories for studying all manner of
astrophysical phenomena.  They are often used as jumping-off points
for investigations of much larger scope and scale; examples are the
searches for extragalactic supernova prompted by the explosion of
SN1987A and the dark matter searches through the technique of
gravitational microlensing.  More than 17,000 papers in the NASA ADS
include the words ``Magellanic Clouds'' in their abstracts or as part
of their keywords, highlighting their importance for a wide variety of
astronomical studies.


An LSST survey that did not include coverage of the Magellanic Clouds
and their periphery would be tragically incomplete.  LSST has a unique
role to play in surveys of the Clouds.  First, its large $A\Omega$
will allow us to probe the thousands of square degrees that comprise
the extended periphery of the Magellanic Clouds with unprecedented
completeness and depth, allowing us to detect and map their extended
disks, stellar halos, and debris from interactions that we already
have strong evidence must exist.  Second, the ability of LSST
to map the entire main bodies in only a few pointings will allow us to
identify and classify their extensive variable source populations with
unprecedented time and areal coverage, discovering, for example,
extragalactic planets, rare variables and transients, and light echoes
from explosive events that occurred thousands of years ago.
Finally, the large number of observing opportunities that the LSST
10-year survey will provide will enable us to produce a static imaging
mosaic of the main bodies of the Clouds with extraordinary image
quality, an invaluable legacy product of LSST.

We have several important scientific questions that can be grouped into two themes, as follows.

\noindent{\bf Galaxy Formation and Evolution}

The study of the formation and
evolution of the Large and Small Magellanic Clouds (LMC and SMC,
respectively), especially their interaction with each other and the
Milky Way. The Magellanic Clouds (MCs) are a unique local laboratory
for studying the formation and evolution of dwarf galaxies in
exquisite detail.  LSST's large FOV will be able to map out the
three-dimensional structure, metallicity and kinematics in great
detail. Within this theme we have three main science questions:
\begin{enumerate}

\item What are the stellar and dark matter mass profiles of the
Magellanic Clouds?  To answer this we need to map their extended disks, halos, debris, and streams.  We can use
streams and RR Lyrae stars as probes of the 3D mass profile.

\item What is the satellite population of the Magellanic Clouds? The
discovery of dwarf satellites by the Dark Energy Survey and other
surveys hint at LSST's potential here.

\item What are the internal dynamics of the Magellanic Clouds?  Proper
motions from HST and from the ground have measured the bulk
motions of the Clouds and have, in combination with spectroscopy,
begun to unravel the three dimensional internal dynamics of the
Clouds.

\end{enumerate}

\noindent{\bf Stellar Astrophysics and Exoplanets}

The MCs have been
used for decades to study stellar astrophysics, microlensing and other
processes.  The fact that the objects are effectively all at a single
known distance makes it much easier to study them than in, for
example, the Milky Way, while the MCs' especial proximity allows us to explore deeper into the luminosity function of the stellar populations. LSST will extend these MC studies to fainter
magnitudes, higher cadence, and larger area. Within this theme we have three main science questions:
\begin{enumerate}

\item How do exoplanet statistics in the Magellanic Clouds compare to
those in the Milky Way?  The calculations in the next section show that
LSST can measure transits of Jupiter-like planets, an intriguing
prospect given the Clouds' lower metallicity environment.

\item What are the variable star and transient population of the Clouds?
LSST will enable population studies, linking star formation and chemical
enrichment histories.

\item What can we learn about supernovae and other explosive events from
their light echoes?  Echoes can give view of such events unavailable by
any other means.

\end{enumerate}


Many different types of objects and measurements with their own
cadence ``requirements'' will fall into these two broad categories
(with some overlap).
% These will be outlined in the next section.
A very important aspect of the ``galaxy evolution'' science theme is
not just the cadence but also the sky coverage of the Magellanic
Clouds ``mini-survey.''  A common misunderstanding is that the MCs
only cover a few degrees on the sky.  That is, however, just the
central regions of the MCs akin to the thinking of the Milky Way as
the just the bulge.  The full galaxies are actually much larger with
LMC stars detected at $\sim$21$^{\circ}$ ($\sim$18 kpc) and SMC stars
at $\sim$10$^{\circ}$ ($\sim$11 kpc) from their respective centers.
The extended stellar debris from their interaction likely extends to
even larger distances.  Therefore, to get a complete picture of the
complex structure of the MCs will require a mini-survey that covers
$\sim$2000 deg$^2$.
% At this point, it not entirely clear how to
% include this into the metrics.
Note, that for the second science case
this is not as much of an issue since the large majority of the
relevant objects will be located in the high-density, central regions
of the MCs.

Our investigation of how the LSST observing strategy will affect the
science outline here is still in its infancy. Some of the disgnostic and
figure of merit metrics developed elsewhere in this paper may be useful
for assessing the Magellanic Cloud science cases as well. In the
meantime we present below two science cases in the early stages of development that show some of the promise LSST shows in this area.

% static science vs. variables
% cadence vs. areal coverage
%  -want to cover whole SCP area to some necessary depth
%  -cadence for time-variable objects in full area or portion
%    what's the minimal # of visits (with decent phase coverage) to do this?  ask Kathy
%    Paula Szkody white paper on doing the full MCs for variables, linked to old MC cadence page
%   maybe a separate metric for spatial structure
%   -variabes:  ability to recover period, classify,


% --------------------------------------------------------------------
%
% \new{Below is a generic list of things we want to measure in the
% Magellanic Clouds.}
% % These will need to be divided among a small set of
% % science sections, each describing a focused science project that has a
% % figure of merit, that is a function of various diagnostic metrics. The
% % final version of the chapter tex file probably will not contain this
% % list.
%
% \begin{enumerate}
%
% \item Deep Color Magnitude Diagrams
% %  -Deep CMDs, just a matter of number of visits
% %  -do the full SMASH (and relevant DES area) with full spatial coverage, at least to SMASH depths, smaller
% %  number of epochs, ~5 sigma at gri~25
% % Knut thoughts: I think we want to make sure that we get 1 mag below old turnoff out to 100 kpc in ugriz with 10sigma precision, i.e. ugriz~25
%
%
% \item Proper Motions
% %-Proper Motions, cadence not as much of an issue, just more epochs
% %  bulk proper motion
% %  LMC spiral motion, streaming motions
% %  internal velocity dispersion
%
%
% %\item Parallaxes
% %-Parallaxes, also mostly a function of nubmer of epochs
% %  bulk distances
% %  internal distance spread
% %  -probably not get parallaxes at MC distances, but could do foreground rejection
%
% \item Variable stars
%
% %-Variables, RR Lyrae, Cepheids might be too bright, dwarf cepheids/scuti good, many more of them.
% %   especially good for getting the 3D structure (out to large distances) of the MCs
% %   -eclipsing binaries (get very accurate distances, see OGLE paper), pulsating WDs, CVs, T Tauri stars
% % novae, supernovae  in Paula's white paper
%
% % use MCs are a way to get templates for variable sources that LSST will detect all over the sky
% % look at variable group metrics
%
%
% \item Transients
% %-Transients, dwarf novae
% % Mike Lund will work on some text for this.  Also did work on cadence considerations
% % for detecting periodic objects in general (periodogram purity function).
%
% \item Transiting Exoplanets
% % -Transiting planets, Mike Lund
% % -need deep drilling to have any hope of finding them
% % -best between ~0.8-1.6 Msun to detect exoplanets, not too faint
% % -can't get ingress/egress but can detect the dip and periodicity
% %   that's enough to characterize
% % -challenging to do follow-up because they are quite faint and too many (?)
% % -cadence?  cover all timescales properly
% % -what's the expected period distribution
%
% % See Lund et al.\ (2015) and the exoplanet discussion in previous part
% % of this paper.
%
% % PASP
%
% %\item Astrometric binaries
% %-Astrometric binaries
%
% \item Light-echoes
% % it's a surface-brightness issue, can see fainter things with LSST than MOSAIC/DECAM
% % could trace out lightcurve if more epochs
%
% \item Gyrochronology
% %-Gychronology, need to get periods of the dwarfs, gives age information
% %  -gyro periods are ~10 days at 1 Gyr and ~30 days at 10 Gyr
%
% \item Interstellar scintillation
% % run in movie mode for 1-2 nights to find missing molecular gas (H2)
% % https://github.com/LSSTScienceCollaborations/ObservingStrategy/issues/68
%
% % Legacy survey
% % use the best seeing to get great data on the MC main bodies
%
% %\item Astroseismology
% %-Astroseismology, dwarfs/giants, giants vary by a couple percent and on "longer" timescales, but
% %    probably too bright for LSST, OGLE probably has best data for those. however LSST might be able to do
% %    asteroseismology of giants to larger distances, measure masses/ages of halo giants!
% %    dwarfs are harder because they vary less and need more higher frequency observations
%
% \end{enumerate}
%
%
% ====================================================================

% PJM: moved to Future Work while MAF analysis is pending:
% % ====================================================================
%+
% SECTION:
%    MCs_ProperMotion.tex
%
% CHAPTER:
%    magclouds.tex
%
% ELEVATOR PITCH:
%-
% ====================================================================

% \section{The Proper Motion of the LMC and SMC}
\subsection{The Proper Motion of the LMC and SMC}
\def\secname{\chpname:propermotion}\label{sec:\secname}

\credit{dnidever},
\credit{knutago}

% SAC Review from Jason Kalirai: This opening paragraph is missing the science hook. There is a clear explanation for how/why LSST proper motions are going to be better than anything before, but the text doesn't actually say what we will learn from such measurements. Is it the case that the resulting constraints on the orbit or past accretion history break some current uncertainty in models of the MC evolution?

In the last decade work with $HST$ has been able to measure the bulk
tangential (in the plane of the sky) velocities ($\sim$300 km/s) of
the Magellanic Clouds (Kallivayalil et al.\ 2016a,b,2013) and even the
rotation of the LMC disk \citep{2014ApJ...781..121V}. Gaia
will measure precise proper motions of stars to $\sim$20th magnitude
which will include the Magellanic red giant branch stars. LSST will be
complementary to Gaia and measure proper motions of stars in the
$\sim$20--24 mag range that includes Magellanic main-sequence stars
which are far more numerous than giants, and, therefore, more useful
for mapping extended stellar structures. The LSST 10-year survey
proper motion precision will be $\sim$0.3--0.4 mas/yr at LMC
main-sequence turnoff at r$\approx$22.5--23.  This will allow for
accurate measurement of proper motions of individual stars at the
$\sim$5$\sigma$ level.

% SAC Review from Jason Kalirai: it's not clear why individual stars are needed. Shouldn't the proper motion precision being referenced here be for the population as a whole?

Besides measuring kinematics, the LSST proper motions can be used to
produce clean samples of Magellanic stars.
In addition, LSST proper motions can be used to improve star/galaxy
separation which is quite significant for faint, blue Magellanic
main-sequence stars.



% streaming motions

% can we do individual LMC stars with LSST, or small groups?

% SRD says want 0.2 mas/yr accuracy over the course of the survey
% 0.2 mas/yr at r=20.5 (similar to Gaia)
% ~0.25 mas/yr at r=22
% ~0.3 mas/yr at r=22.5 LMC turnoff
% ~0.4 mas/yr at r=23
% 1 mas/yr for r=24
% See Figure 21 from Ivezic et al. (2012) or slide 46 of overview-sci-reqs.pdf
%have the astrometric precision to measure the proper motions of individual
%Proper motion cleaning to find the giants?? gaia does that already
%lsst can use proper motion cleaning to do star/galaxy separation as well
%
%The
%The Magellanic Clouds have a large tangential velocity (in the plane of the sky) that has been
%Gaia will be able to see the bright stuff, need lsst to get the MSTO
%gaia/lsst synergy
%
%metric that calculates the proper motion accuracy of LMC MSTO stars at r=23 and calculates the sigma-level of
%the proper motion measurement (2.0 mas/yr / sigma_pm ).

% --------------------------------------------------------------------

% \subsection{Metrics}
\subsubsection{Metrics}
\label{sec:\chpname:metrics}

The natural Figure of Merit for this science case is the precision
with which the proper motion of the Magellanic Clouds can be measured.
This is likely to depend on the following diagnostic metrics:

% metric on surface brightness limit in different parts of the sky
% to MC structure
% could use metric for how much of the Besla stellar debris we can detect
%  even just that region of the sky covered

\begin{itemize}

\item  Single star proper motion precision, possibly quantified as.
the significance level
($\sigma$-level) of the proper motion measurement of one Magellanic MSTO
star (r=23 mag). We would expect this to take values of
$\sim$2.0 mas yr$^{-1}$ / $\sigma_{\rm pm}$.
%  the precision proper motion metric that
%  Metric that calculates the proper motion accuracy of LMC MSTO stars at r=23 and calculates the sigma-level of
%the proper motion measurement (2.0 mas/yr / sigma_pm ).

\item Another useful diagnostic metric would be the surface
brightness limit of the Magellanic structures, using MSTO stars.

\item A metric quantifying how much of the expected Magellanic debris/structure
\citet[from][]{2012MNRAS.421.2109B} model) we can detect would allow the
proper motion science case to be extended to peripheral structures.
This would depend
mostly on the area covered, but we could also use the surface
brightness limit (calculated above) directly.

\end{itemize}

% % --------------------------------------------------------------------
%
% \subsection{Metrics}
% \label{sec:\secname:metrics}
%
% % --------------------------------------------------------------------
%
% \subsection{OpSim Analysis}
% \label{sec:\secname:analysis}
%
% % --------------------------------------------------------------------
%
% \subsection{Discussion}
% \label{sec:\secname:discussion}
%
% ====================================================================
%
% \subsection{Conclusions}
%
% Here we answer the ten questions posed in
% \autoref{sec:intro:evaluation:caseConclusions}:
%
% \begin{description}
%
% \item[Q1:] {\it Does the science case place any constraints on the
% tradeoff between the sky coverage and coadded depth? For example, should
% the sky coverage be maximized (to $\sim$30,000 deg$^2$, as e.g., in
% Pan-STARRS) or the number of detected galaxies (the current baseline
% of 18,000 deg$^2$)?}
%
% \item[A1:] ...
%
% \item[Q2:] {\it Does the science case place any constraints on the
% tradeoff between uniformity of sampling and frequency of  sampling? For
% example, a rolling cadence can provide enhanced sample rates over a part
% of the survey or the entire survey for a designated time at the cost of
% reduced sample rate the rest of the time (while maintaining the nominal
% total visit counts).}
%
% \item[A2:] ...
%
% \item[Q3:] {\it Does the science case place any constraints on the
% tradeoff between the single-visit depth and the number of visits
% (especially in the $u$-band where longer exposures would minimize the
% impact of the readout noise)?}
%
% \item[A3:] ...
%
% \item[Q4:] {\it Does the science case place any constraints on the
% Galactic plane coverage (spatial coverage, temporal sampling, visits per
% band)?}
%
% \item[A4:] ...
%
% \item[Q5:] {\it Does the science case place any constraints on the
% fraction of observing time allocated to each band?}
%
% \item[A5:] ...
%
% \item[Q6:] {\it Does the science case place any constraints on the
% cadence for deep drilling fields?}
%
% \item[A6:] ...
%
% \item[Q7:] {\it Assuming two visits per night, would the science case
% benefit if they are obtained in the same band or not?}
%
% \item[A7:] ...
%
% \item[Q8:] {\it Will the case science benefit from a special cadence
% prescription during commissioning or early in the survey, such as:
% acquiring a full 10-year count of visits for a small area (either in all
% the bands or in a  selected set); a greatly enhanced cadence for a small
% area?}
%
% \item[A8:] ...
%
% \item[Q9:] {\it Does the science case place any constraints on the
% sampling of observing conditions (e.g., seeing, dark sky, airmass),
% possibly as a function of band, etc.?}
%
% \item[A9:] ...
%
% \item[Q10:] {\it Does the case have science drivers that would require
% real-time exposure time optimization to obtain nearly constant
% single-visit limiting depth?}
%
% \item[A10:] ...
%
% \end{description}

% ====================================================================
%
% \navigationbar


% ====================================================================

\navigationbar


% ====================================================================
%+
% SECTION:
%    MCs_FutureWork.tex
%
% CHAPTER:
%    transients.tex
%
% ELEVATOR PITCH:
%    Ideas for future metric investigation, with quantitaive analysis
%    still pending.
%-
% ====================================================================

\section{Future Work}
\def\secname{\chpname:future}\label{sec:\secname}

In this section we provide a short compendium of science cases that
are either still being developed, or that are deserving of quantitative
MAF analysis at some point in the future.

% ====================================================================

% ====================================================================
%+
% SECTION:
%    MCs_ProperMotion.tex
%
% CHAPTER:
%    magclouds.tex
%
% ELEVATOR PITCH:
%-
% ====================================================================

% \section{The Proper Motion of the LMC and SMC}
\subsection{The Proper Motion of the LMC and SMC}
\def\secname{\chpname:propermotion}\label{sec:\secname}

\credit{dnidever},
\credit{knutago}

% SAC Review from Jason Kalirai: This opening paragraph is missing the science hook. There is a clear explanation for how/why LSST proper motions are going to be better than anything before, but the text doesn't actually say what we will learn from such measurements. Is it the case that the resulting constraints on the orbit or past accretion history break some current uncertainty in models of the MC evolution?

In the last decade work with $HST$ has been able to measure the bulk
tangential (in the plane of the sky) velocities ($\sim$300 km/s) of
the Magellanic Clouds (Kallivayalil et al.\ 2016a,b,2013) and even the
rotation of the LMC disk \citep{2014ApJ...781..121V}. Gaia
will measure precise proper motions of stars to $\sim$20th magnitude
which will include the Magellanic red giant branch stars. LSST will be
complementary to Gaia and measure proper motions of stars in the
$\sim$20--24 mag range that includes Magellanic main-sequence stars
which are far more numerous than giants, and, therefore, more useful
for mapping extended stellar structures. The LSST 10-year survey
proper motion precision will be $\sim$0.3--0.4 mas/yr at LMC
main-sequence turnoff at r$\approx$22.5--23.  This will allow for
accurate measurement of proper motions of individual stars at the
$\sim$5$\sigma$ level.

% SAC Review from Jason Kalirai: it's not clear why individual stars are needed. Shouldn't the proper motion precision being referenced here be for the population as a whole?

Besides measuring kinematics, the LSST proper motions can be used to
produce clean samples of Magellanic stars.
In addition, LSST proper motions can be used to improve star/galaxy
separation which is quite significant for faint, blue Magellanic
main-sequence stars.



% streaming motions

% can we do individual LMC stars with LSST, or small groups?

% SRD says want 0.2 mas/yr accuracy over the course of the survey
% 0.2 mas/yr at r=20.5 (similar to Gaia)
% ~0.25 mas/yr at r=22
% ~0.3 mas/yr at r=22.5 LMC turnoff
% ~0.4 mas/yr at r=23
% 1 mas/yr for r=24
% See Figure 21 from Ivezic et al. (2012) or slide 46 of overview-sci-reqs.pdf
%have the astrometric precision to measure the proper motions of individual
%Proper motion cleaning to find the giants?? gaia does that already
%lsst can use proper motion cleaning to do star/galaxy separation as well
%
%The
%The Magellanic Clouds have a large tangential velocity (in the plane of the sky) that has been
%Gaia will be able to see the bright stuff, need lsst to get the MSTO
%gaia/lsst synergy
%
%metric that calculates the proper motion accuracy of LMC MSTO stars at r=23 and calculates the sigma-level of
%the proper motion measurement (2.0 mas/yr / sigma_pm ).

% --------------------------------------------------------------------

% \subsection{Metrics}
\subsubsection{Metrics}
\label{sec:\chpname:metrics}

The natural Figure of Merit for this science case is the precision
with which the proper motion of the Magellanic Clouds can be measured.
This is likely to depend on the following diagnostic metrics:

% metric on surface brightness limit in different parts of the sky
% to MC structure
% could use metric for how much of the Besla stellar debris we can detect
%  even just that region of the sky covered

\begin{itemize}

\item  Single star proper motion precision, possibly quantified as.
the significance level
($\sigma$-level) of the proper motion measurement of one Magellanic MSTO
star (r=23 mag). We would expect this to take values of
$\sim$2.0 mas yr$^{-1}$ / $\sigma_{\rm pm}$.
%  the precision proper motion metric that
%  Metric that calculates the proper motion accuracy of LMC MSTO stars at r=23 and calculates the sigma-level of
%the proper motion measurement (2.0 mas/yr / sigma_pm ).

\item Another useful diagnostic metric would be the surface
brightness limit of the Magellanic structures, using MSTO stars.

\item A metric quantifying how much of the expected Magellanic debris/structure
\citet[from][]{2012MNRAS.421.2109B} model) we can detect would allow the
proper motion science case to be extended to peripheral structures.
This would depend
mostly on the area covered, but we could also use the surface
brightness limit (calculated above) directly.

\end{itemize}

% % --------------------------------------------------------------------
%
% \subsection{Metrics}
% \label{sec:\secname:metrics}
%
% % --------------------------------------------------------------------
%
% \subsection{OpSim Analysis}
% \label{sec:\secname:analysis}
%
% % --------------------------------------------------------------------
%
% \subsection{Discussion}
% \label{sec:\secname:discussion}
%
% ====================================================================
%
% \subsection{Conclusions}
%
% Here we answer the ten questions posed in
% \autoref{sec:intro:evaluation:caseConclusions}:
%
% \begin{description}
%
% \item[Q1:] {\it Does the science case place any constraints on the
% tradeoff between the sky coverage and coadded depth? For example, should
% the sky coverage be maximized (to $\sim$30,000 deg$^2$, as e.g., in
% Pan-STARRS) or the number of detected galaxies (the current baseline
% of 18,000 deg$^2$)?}
%
% \item[A1:] ...
%
% \item[Q2:] {\it Does the science case place any constraints on the
% tradeoff between uniformity of sampling and frequency of  sampling? For
% example, a rolling cadence can provide enhanced sample rates over a part
% of the survey or the entire survey for a designated time at the cost of
% reduced sample rate the rest of the time (while maintaining the nominal
% total visit counts).}
%
% \item[A2:] ...
%
% \item[Q3:] {\it Does the science case place any constraints on the
% tradeoff between the single-visit depth and the number of visits
% (especially in the $u$-band where longer exposures would minimize the
% impact of the readout noise)?}
%
% \item[A3:] ...
%
% \item[Q4:] {\it Does the science case place any constraints on the
% Galactic plane coverage (spatial coverage, temporal sampling, visits per
% band)?}
%
% \item[A4:] ...
%
% \item[Q5:] {\it Does the science case place any constraints on the
% fraction of observing time allocated to each band?}
%
% \item[A5:] ...
%
% \item[Q6:] {\it Does the science case place any constraints on the
% cadence for deep drilling fields?}
%
% \item[A6:] ...
%
% \item[Q7:] {\it Assuming two visits per night, would the science case
% benefit if they are obtained in the same band or not?}
%
% \item[A7:] ...
%
% \item[Q8:] {\it Will the case science benefit from a special cadence
% prescription during commissioning or early in the survey, such as:
% acquiring a full 10-year count of visits for a small area (either in all
% the bands or in a  selected set); a greatly enhanced cadence for a small
% area?}
%
% \item[A8:] ...
%
% \item[Q9:] {\it Does the science case place any constraints on the
% sampling of observing conditions (e.g., seeing, dark sky, airmass),
% possibly as a function of band, etc.?}
%
% \item[A9:] ...
%
% \item[Q10:] {\it Does the case have science drivers that would require
% real-time exposure time optimization to obtain nearly constant
% single-visit limiting depth?}
%
% \item[A10:] ...
%
% \end{description}

% ====================================================================
%
% \navigationbar


% ====================================================================

% ====================================================================
%+
% SECTION:
%    MCs_ProperMotion.tex
%
% CHAPTER:
%    magclouds.tex
%
% ELEVATOR PITCH:
%-
% ====================================================================

% \section{The Proper Motion of the LMC and SMC}
\subsection{Exoplanets in the LMC and SMC}
\def\secname{\chpname:MC_exoplanets}\label{sec:\secname}

\credit{lundmb},
\credit{migueldvb}

% SAC Review by Jason Kalirai: I was confused about the part of this section related to finding transiting exoplanets in LMC stars. First, it would be nice to show some of the analysis in the paper itself, hopefully backed up by simulations of LSST's performance. Second, if the motivation is to tackle this in the Clouds due to their low metallicity, why not simply propose for such an experiment in a more nearby metal-poor system (with or without LSST).

While exoplanets are discussed in greater depth in \autoref{sec:planets}, it is
also worth noting here the unique circumstance of exoplanets in the
Magellanic Clouds. To date, all detected exoplanets have been found around
host stars within the Milky Way. Any constraints that could be applied to
planet occurrence rates in such a different stellar population as is found
in the Magellanic Clouds would provide a fresh insight into the limits
that are to be placed on planet formation rates.

The transit method of exoplanet detection is constrained by sufficient
period coverage in the observations taken, and in the dimming caused by
the star's transit being large enough with respect to the noise in
observations that the periodic signal of the transit can be recovered.
The relatively small chance of a planet being present and properly
aligned is offset by observing a large number of stars simultaneously.
Simulations have already shown that LSST has the capability to recover
the correct periods for large exoplanets around stars at the distance of
the LMC \citet{2015AJ....149...16L}.  We note that it would be unlikely
to be able to conduct follow-up observations of the discovered
candidates to confirm their planetary nature at a distance of $\sim$
50 kpc.  Further work is needed to characterize the ability to detect
these planets with sufficiently significant power to determine the
planet yield that could be expected from the LMC (Lund et al. in prep).


% --------------------------------------------------------------------

% \subsection{Metrics}
\subsubsection{Metrics}
\label{sec:\secname:metrics}

The case of transiting exoplanets in the Magellanic Clouds will benefit
from the same metrics that are used by transiting exoplanets within the
Milky Way, and are addressed in \autoref{sec:variables:variablemetrics}
and \autoref{sec:planets}. The key properties of the OpSim to be
measured will be those that relate to the number of observations that
will be made during planetary transits, and the overall phase coverage
of observations.  Unlike the general case of transiting planets in LSST,
transiting planets in the Magellanic Clouds specifically will likely
only have any meaning in deep-drilling fields, or some other comparable
cadence.

% % --------------------------------------------------------------------
%
% \subsection{Metrics}
% \label{sec:\secname:metrics}
%
% % --------------------------------------------------------------------
%
% \subsection{OpSim Analysis}
% \label{sec:\secname:analysis}
%
% % --------------------------------------------------------------------
%
% \subsection{Discussion}
% \label{sec:\secname:discussion}
%
% ====================================================================
%
% \subsection{Conclusions}
%
% Here we answer the ten questions posed in
% \autoref{sec:intro:evaluation:caseConclusions}:
%
% \begin{description}
%
% \item[Q1:] {\it Does the science case place any constraints on the
% tradeoff between the sky coverage and coadded depth? For example, should
% the sky coverage be maximized (to $\sim$30,000 deg$^2$, as e.g., in
% Pan-STARRS) or the number of detected galaxies (the current baseline
% of 18,000 deg$^2$)?}
%
% \item[A1:] ...
%
% \item[Q2:] {\it Does the science case place any constraints on the
% tradeoff between uniformity of sampling and frequency of  sampling? For
% example, a rolling cadence can provide enhanced sample rates over a part
% of the survey or the entire survey for a designated time at the cost of
% reduced sample rate the rest of the time (while maintaining the nominal
% total visit counts).}
%
% \item[A2:] ...
%
% \item[Q3:] {\it Does the science case place any constraints on the
% tradeoff between the single-visit depth and the number of visits
% (especially in the $u$-band where longer exposures would minimize the
% impact of the readout noise)?}
%
% \item[A3:] ...
%
% \item[Q4:] {\it Does the science case place any constraints on the
% Galactic plane coverage (spatial coverage, temporal sampling, visits per
% band)?}
%
% \item[A4:] ...
%
% \item[Q5:] {\it Does the science case place any constraints on the
% fraction of observing time allocated to each band?}
%
% \item[A5:] ...
%
% \item[Q6:] {\it Does the science case place any constraints on the
% cadence for deep drilling fields?}
%
% \item[A6:] ...
%
% \item[Q7:] {\it Assuming two visits per night, would the science case
% benefit if they are obtained in the same band or not?}
%
% \item[A7:] ...
%
% \item[Q8:] {\it Will the case science benefit from a special cadence
% prescription during commissioning or early in the survey, such as:
% acquiring a full 10-year count of visits for a small area (either in all
% the bands or in a  selected set); a greatly enhanced cadence for a small
% area?}
%
% \item[A8:] ...
%
% \item[Q9:] {\it Does the science case place any constraints on the
% sampling of observing conditions (e.g., seeing, dark sky, airmass),
% possibly as a function of band, etc.?}
%
% \item[A9:] ...
%
% \item[Q10:] {\it Does the case have science drivers that would require
% real-time exposure time optimization to obtain nearly constant
% single-visit limiting depth?}
%
% \item[A10:] ...
%
% \end{description}
%
% ====================================================================
%
% \navigationbar


% ====================================================================

\navigationbar


% ====================================================================
